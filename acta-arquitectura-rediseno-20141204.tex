% Created 2014-12-04 Thu 15:38
\documentclass[11pt]{article}
\usepackage[utf8]{inputenc}
\usepackage[T1]{fontenc}
\usepackage{fixltx2e}
\usepackage{graphicx}
\usepackage{longtable}
\usepackage{float}
\usepackage{wrapfig}
\usepackage{rotating}
\usepackage[normalem]{ulem}
\usepackage{amsmath}
\usepackage{textcomp}
\usepackage{marvosym}
\usepackage{wasysym}
\usepackage{amssymb}
\usepackage{hyperref}
\tolerance=1000
\usepackage[spanish]{babel}
\author{Development Ventura24}
\date{\today}
\title{Reunión Cambios BE por el Rediseño}
\hypersetup{
  pdfkeywords={},
  pdfsubject={},
  pdfcreator={Emacs 24.3.1 (Org mode 8.2.6)}}
\begin{document}

\maketitle
\tableofcontents

\section{Introducción}
\label{sec-1}
De cara al próximo rediseño es conveniente estudiar y discutir la oportunidad de realizar cambios en la plataforma, en cuanto a tecnologías de back-end.
Con ese objeto se ha realizado una reunión el día 04/12/2014, con los siguientes asistentes:
\begin{itemize}
\item Sergio García-Mejía
\item Jose Luís Sánchez
\item Ramón Crespo
\item Jose Juan
\item Jose San Leandro
\end{itemize}
\section{Análisis de la solución actual}
\label{sec-2}
Se ha comenzado analizando y debatiendo la solución actual, enfocándonos inicialmente en sus puntos fuertes y puntos débiles.
El objetivo de este análisis es servir de marco comparativo para poder dotar de cierto rigor las futuras propuestas.
\subsection{Puntos fuertes}
\label{sec-2-1}
\begin{itemize}
\item Estabilidad
\item Gestión de integraciones
\item Se basa en tecnologías conocidas, no propietarias
\item Integración con un sistema CMS
\item Capacidad de adaptación a los requisitos de www.ventura24.es
\end{itemize}
\subsection{Puntos débiles}
\label{sec-2-2}
\begin{itemize}
\item Mantenimiento y estado actual de los JSPs
\item Mapping de URLs opaco (no se puede conocer rápida ni fácilmente, a partir de una URL, su action y jsp)
\item Redirecciones repartidas en Java y Apache
\item Rendimiento de los actions: WebAppAction se ejecuta múltiples veces en cada request
\item Se basa en Struts 1, que es una tecnología de hace 10 años.
\item Poco flexible en contextos diferentes a www.ventura24.es
\item Las validaciones con struts-validator son incómodas de implementar, y frágiles
\end{itemize}
\section{Pruebas de concepto}
\label{sec-3}
Se han clasificado los distintos enfoques en función de una estimación a priori de envergadura de los cambios respecto a la solución actual.
Los requisitos que deben cumplir las propuestas son:
\begin{itemize}
\item Requisitos funcionales: implementar /, /login.do, /loginerror.do, /depot.do
\item Requisitos no funcionales:
\begin{itemize}
\item soportar integraciones.
\item rendimiento equivalente.
\end{itemize}
\end{itemize}

Se creará un repositorio PoC para hospedar las propuestas.

En la reunión se ha decidido clasificar las mismas de acuerdo a su impacto:
\subsection{Propuestas conservadoras}
\label{sec-3-1}
En ese grupo se contemplan las propuestas que a priori conllevan menos coste o riesgo, pero que a la vez mejorarían la deuda técnica, la mantenibilidad, y la flexibilidad.
Serían cambios que afectan únicamente al módulo nlp-webapp.
Las propuestas dentro de este grupo son:
\subsubsection{Refactoring Struts 1}
\label{sec-3-1-1}
\begin{itemize}
\item Responsable: Jose San Leandro
\item Objetivo:
\begin{itemize}
\item Definir una propuesta de cómo hacer el mapping de urls y organizar los JSPs.
\item proponer cómo implementar los JSPs, tiles y jsp:include, etc.
\end{itemize}
\end{itemize}
\subsection{Propuestas intermedias}
\label{sec-3-2}
En esta categoría tienen cabida aquellas propuestas que impliquen un cambio de framework en la capa de presentación (nlp-webapp).
Cada propuesta debe servir para contrastar las características de la solución propuesta, respecto a los puntos definidos como base.
\subsubsection{Play}
\label{sec-3-2-1}
\begin{itemize}
\item Responsable: Ramón Crespo
\item Objetivo:
\begin{itemize}
\item Implementar la funcionalidad de las urls citadas con Play.
\item Describir la solución (mapeo de urls, gestión de las vistas, etc).
\end{itemize}
\end{itemize}
\subsubsection{Spring MVC}
\label{sec-3-2-2}
\begin{itemize}
\item Responsable: José Luís Sánchez
\item Objetivo:
\begin{itemize}
\item Implementar la funcionalidad de las urls citadas con Spring MVC.
\item Describir la solución (mapeo de urls, gestión de las vistas, etc).
\end{itemize}
\end{itemize}
\subsection{Propuestas más agresivas}
\label{sec-3-3}
Esta categoría incluye aquellas propuestas que impliquen cambios más significativos en la arquitectura, no sólo nlp-webapp.
\subsubsection{TBD}
\label{sec-3-3-1}
No hay propuestas por el momento.
\section{Siguientes pasos}
\label{sec-4}
El jueves 18/12/2014 se expondrán las soluciones propuestas de cara a su evaluación y discusión.
% Emacs 24.3.1 (Org mode 8.2.6)
\end{document}
